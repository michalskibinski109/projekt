\documentclass[a4paper,12pt]{scrartcl}
\usepackage{hyperref}
\usepackage{url}            % simple URL typesetting
\usepackage{booktabs}       % professional-quality tables
\usepackage{amsfonts}       % blackboard math symbols
\usepackage{amsmath}
\usepackage{nicefrac}       % compact symbols for 1/2, etc.
\usepackage{microtype}      % microtypography
\usepackage{tabto}
\usepackage{tikz}
\usepackage{graphicx}
\usepackage{longtable}
\usepackage{lmodern}
\usepackage{listings}
\usepackage[T1]{fontenc}
\usepackage[utf8]{inputenc}
\usepackage{forest}
\definecolor{folderbg}{RGB}{124,166,198}
\definecolor{folderborder}{RGB}{110,144,169}

\def\Size{4pt}
\tikzset{
  folder/.pic={
    \filldraw[draw=folderborder,top color=folderbg!50,bottom color=folderbg]
      (-1.05*\Size,0.2\Size+5pt) rectangle ++(.75*\Size,-0.2\Size-5pt);  
    \filldraw[draw=folderborder,top color=folderbg!50,bottom color=folderbg]
      (-1.15*\Size,-\Size) rectangle (1.15*\Size,\Size);
  }
}




\title{ROZPOZNAWANIE KOTKÓW OD PIESKÓW}
\author{Michał Skibiński (260352)}
\date{30.05.2022}

\begin{document}

\maketitle

\section{Wstęp}
Przedmiotem projektu jest zadanie klasyfikacji binarnej (klasyfikacji kotów oraz psów).
do rozwiązania problemu posłużyłem się \textbf{konwolucyjną siecią neuronową (CNN)}.
Ponieważ sieci neuronowe operują na macierzach, są one szczególnie wydajne przy pracy z obrazami, 
które tak naprawde są macierzami pikseli. \\
Rozwiązanie zaprojektowałem  w języku \textit{python} używając biblioteki \textit{tensorflow}.\\
Zdecydowałem się na język programowania \textit{python}, ze względu na jego prostotę, 
niezawodność oraz wielorakość bibliotek ogólno-dostępnych. \\
Bibliotekę \textit{tensorflow} wybrałem ze względu na możliwość 
wykonywania złożonych operacji na sieciach neuronowych, przy użyciu stosunkowo
niewielkiej ilości kodu. Oprócz tego biblioteka ta działa bardzo efektywnie,
 oraz została do niej napisana bogata dokumentacja.\\

\newpage{}
\section{Model}


\subsection{przygotowanie danych}
aby przygotować dane kolejno:
\begin{enumerate}

    \item napisałem algorytm do odczytywania zdjęć z zadanego folderu i tworzeniu folderu plików z podziałem na \textbf{dane testowe i treningowe} oraz klasy, tak by strukrura wyglądała następująco: \\
    \begin{forest}
        for tree={
          font=\ttfamily,
          grow'=0,
          child anchor=west,
          parent anchor=south,
          anchor=west,
          calign=first,
          inner xsep=7pt,
          edge path={
            \noexpand\path [draw, \forestoption{edge}]
            (!u.south west) +(7.5pt,0) |- (.child anchor) pic {folder} \forestoption{edge label};
          },
          before typesetting nodes={
            if n=1
              {insert before={[,phantom]}}
              {}
          },
          fit=band,
          before computing xy={l=15pt},
        }  
      [input for model
        [train
          [cats (3830 zdjęć)
          ]
          [dogs (1915 zdjęć)
          ]
        ]
        [test
            [cats (958 zdjęć)
            ]
            [dogs (479 zdjęć)
            ]
        ]
      ]
      \end{forest}\\
      ilość danych testowych ustawiłem na 20 procent.
    \item usunąłem różnicę w ilości fotografii kotów i psów (ważne aby sieć neuronowa miała tyle samo danych z każdej z klas) 
    \item dokonałem \textbf{normalizacji danych} - przeskalowałem wartości pikseli tak by pochodziły z zakresu [0,1)
    \item ustandaryzowałem dane - wszystkie zdjęcia przeskalowałem do rozmiarów [224 x 224] px.
    \item zrezygnowałem z użycia \textbf{PCA} ponieważ redukcja wymiarów wiązała się ze zbyt dużą utratą danych. 
      
  \end{enumerate}
\subsection{budowa modelu}
w fazie budowania modelu, inspirowałem się dotychczasowymi modelami do rozwiązania problemu.
Zanim pojawił się finałowy model stworzyłem ok 10 innych modeli, każdy był usprawnieniem poprzedniego.
skupiłem się na \textbf{funkcjach aktywujących} dla poszczególnych warstw oraz liczbie ich liczbie.\\\\

włsności modelu:
\begin{itemize}
  \item wejście dla modelu: macierz o rozmiarach (224 x 224 x 3) - gdzie 3 wynika z reprezentacji pikseli poprzez RGB 
  \item liczba warstw ukrytych: 20 
  \item wyjście: wartość 0 lub 1 (0 - dla psa, 1 - dla kota) 
  \item metoda optymalizacji:  \textbf{stochastyczny spadek gradientu(SGD)} z prędkością nauki = 0.0001
  \item funkcja strat:  binarna entropia krzyżowa
  \item ilość generacji(epok):  20
  \item populacja w każdej generacji: 60
  \item \textbf{funkcja regularyzacji}: l2 (bez regularyzacji występowało \textbf{przetrenowanie modelu})
\end{itemize}  

\subsection{przebieg treningu}

\begin{figure}[h]
  \includegraphics[width=\linewidth]{TRENINNG.png}
\end{figure}
\newpage{}
\section{Wyniki}
\subsection{jakość modelu}
po skończonym treningu precyzja:
\begin{itemize}
  \item dla danych treningowych wynosiosła: \textbf{98 procent}
  \item dla danych testowych: \textbf{93 procent}

\end{itemize}  

\begin{figure}[h]
  \includegraphics[width=\linewidth]{preedictions.png}
  aby zweryfikować poprawność modelu uruchomiłem go dla zdjęć zrobionych przeze mnie oraz moich znajomych,
  tak aby mieć pewność że zdjęcia są różnych formatów, rozmiarów, różnej jakości i nie 
  były wcześniej w żaden sposób przerobione
\end{figure}
\newpage{}
\subsection{przyczyny takiej a nie innej jakość modelu}
model okazał się stosunkowo skuteczny, przy klasyfikacji 
zdjęć nie spreparowanych skuteczność na poziomie 93 
procent, możemy uznać za zadowalającą.\\\\
\textbf{dlaczego model nie ma skutecznośćci na poziomie 100 procent?}\\
\begin{enumerate}
  \item ponieważ dysponowałem urządzeniem o niewielkiej możliwości obliczeniowej,
   a operacje na zdjęciach, są skomplikowane, trenowanie sieci skończyłem gdy jej dokładość miała tendencje 
   rosnące.
  \item niektóre ze zdjęć mogły być nieostre, lub przedstawiać zwierzę które jest mieszanką genetyczną psa i kota.
  przykład takich zdjęć: 
  \begin{figure}[h]
    \includegraphics[width=\linewidth]{example.png}
  \end{figure}
\end{enumerate} 



\section{przypisy}
zdjęcia użyte do budowy dla modelu:\\ 
\href{https://www.kaggle.com/datasets/zippyz/cats-and-dogs-breeds-classification-oxford-dataset}
{https://www.kaggle.com/datasets/zippyz/cats-and-dogs-breeds-classification-oxford-dataset}
kod rozwiązania:\\
\href{https://github.com/michalskibinski109/projekt/blob/main/PROJEKT.ipynb}
{https://github.com/michalskibinski109/projekt/blob/main/PROJEKT.ipynb}
\end{document}

